\documentclass{book}
\usepackage[latin1]{inputenc}
\usepackage[T1]{fontenc}
\usepackage[acronym]{glossaries}

\makeglossaries

\newglossaryentry{latex}
{
        name=latex,
        description={Is a mark up language specially suited for 
scientific documents}
}

\newglossaryentry{maths}
{
        name=mathematics,
        description={Mathematics is what}
}

\newglossaryentry{formula}
{
        name=formula,
        description={A mathematical expression}
}

\newacronym{gcd}{GCD}{Greatest Common Divisor}

\newacronym{lcm}{LCM}{Least Common Multiple}
\begin{document}
\tableofcontents
\part{Partie 1}
    \chapter{Chapitre 1}
        \section*{Une section}
        \section{Technical Information}
        The first part of the document should provide information about the following items :
        \begin{itemize}
  			\item \textbf{\underline{Deadline/Due Date :}} Provided by the business, this limit give information to the developer of the time allowed for the new feature.
  			\item Individual entries are indicated with a black dot, a so-called bullet.
  			\item The text in the entries may be of any length.
		\end{itemize}
        \section{Problem Statement}
            qsdsqd
        \subsection{Validation Process}
        In this section, business/product owner/product manager/QA should provide 
        \subsection{Une sous-section}
    \chapter{Chapitre 2}
        \section{Une section}
            \subsection{Une sous-section}
                \paragraph{Un paragraphe}
            \subsection{Une sous-section}
                \paragraph{Un paragraphe}
        \section{Une section}
    \chapter{Chapitre 3}
        \section{Une section}
        \section{Une section}
      
  qdqsdqsd  

The \Gls{latex} typesetting markup language is specially suitable 
for documents that include \gls{maths}. \Glspl{formula} are rendered 
properly an easily once one gets used to the commands.

\printglossary[type=\acronymtype]

\printglossary[nonumberlist]
\end{document}